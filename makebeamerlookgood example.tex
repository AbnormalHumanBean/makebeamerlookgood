%%%%%%%%%%%%%%%%%%%%%%%%%%%%%%
% Example for makebeamerbetter 

\documentclass[aspectratio=1610]{beamer}
% Aspect ratio can be changed here. It is set to my abnormal screen size. 

\usepackage{graphicx} 
\usepackage{xcolor}

\usepackage{makebeamerlookgood}
% The package itself(Currently must be installed on own system)

%%%%%%%%%%%%%%%%%%%%%%%%%%%%%%%%%%%%%%%%%%%%%%%%%%%%%%%%%%%%%%%%%%%%%%%%%%%%%%
\usepackage{amssymb}
\usepackage{pifont}
\usepackage{adfbullets}
\usepackage{fontawesome}

% For many different bullet points
%%%%%%%%%%%%%%%%%%%%%%%%%%%%%%%%%%%%%%%%%%%%%%%%%%%%
\usepackage{soul} 
\usepackage{tcolorbox,mathtools}
\usepackage{tikz}
% For creating highlights and annotating eqautions

\usepackage{annotate-equations}
\usetikzlibrary{arrows.meta, calc, quotes, tikzmark}

%%%%%%%%%%%%%%%%%%%%%%%%%%%%%% 
% For creating high lights. Can change to any color
\definecolor{lightblue}{rgb}{.90,.95,1}
\sethlcolor{yellow}
\renewcommand<>{\hl}[1]{\only#2{\beameroriginal{\hl}}{#1}}
\makeatletter
\newcommand\SoulColor{%
	\let\set@color\beamerorig@set@color
	\let\reset@color\beamerorig@reset@color}
\makeatother
\SoulColor
%%%%%%%%%%%%%%%%%%%%%%%%%%%%
% Here is the information needed for the title page 
\title{Title: With super long subtitle because economists can pick short titles}
\author{Your Name}
\date{\today}
\institute{Texas Tech University}
%%%%%%%%%%%%%%%%%%%%%%%%%%%%%%

%%%%%%%%%%%%%%%%%%%%%%%%%%%%%%%%%%%%%%%%%%%%%%%%%%%
% Optional Font changes

%\usepackage{sourcecodepro}


%\usepackage[mathrm=sym]{unicode-math}
%\setmathfont{Fira Math}

%%%%%%%%%%%%%%%%%%%%%%%%%%%%%%%%%%%%%%%%%%%%%%%%%%%%
\begin{document}

%to change the annonations of equations
	\tikzset{annotate equations/arrow/.style={<-}}
	\tikzset{annotate equations/arrow/.style=red}
	
	% To make a new item type for that specfic item
	\newcommand{\then}{\item[\faArrowCircleRight]}
	
	\begin{frame}[plain]
		\hspace*{-.75\beamersidebarwidth}\begin{minipage}{\textwidth}	\large{\inserttitle} \\
			\textcolor{ttu_red}{\dotfill}	\\
			\normalsize{\insertauthor} \\
			\normalsize{\insertinstitute}
		\end{minipage}
	\end{frame}
	
\section{Here}
	
	\begin{frame}{Title of Slide}
		\begin{itemize}
			\setlength\itemsep{1em}
			\item[\ding{108}] an Item 
			\item[\ding{108}] another item  
			\begin{itemize}
				\setlength\itemsep{.75em}
				\item[\ding{108}] These are under those
				\item[\ding{108}] Here's some math $Corr(\varepsilon ,\mu) \neq 0$ 
				\end{itemize}
		\end{itemize}
	\end{frame}
	
\section{Examples}
\subsection{Table}
	
	\begin{frame}{Example Table}
		\begin{table}
			{\renewcommand{\arraystretch}{1.4}%
				\begin{tabular}{l}
					\hline
					\hline
					\strong{Headline} \\
					\hline	
				a line  \\
			a new line \\
			this is what it looks it like \\
			Another item \\
					\hline
					\hline
			\end{tabular}}
		\end{table}
	\end{frame}
	
\subsection{Math Example}

\begin{frame}{Math}
	\begin{equation*}
		\eqnmark{c}{B_{ijt}} = \alpha + \beta \eqnmark{d}{\mathbb{I} [VI \times supplies\ owner \ firm]_{ijt}} + \gamma_{ij} + \delta_t + \varepsilon_{ijt} \tag{3}
	\end{equation*}
	\visible<2>{\annotate[yshift=1em]{above}{c}{this here}}
	\visible<2>{\annotate[yshift =-1em]{below}{d}{This here}}
\end{frame}

\section{Conclusion}

\begin{frame}{Conclusion}
	
	\begin{itemize}
		\then Conclusion 1
		\then Conclusion 2 
		\then Conclusion 3
		\then Conclusion 4
	\end{itemize}
	
\end{frame}

\end{document}